\documentclass[11pt,letterpaper]{article}

% Packages
\usepackage[margin=1in]{geometry}
\usepackage{amsmath,amsfonts,amssymb}
\usepackage{graphicx}
\usepackage{xcolor}
\usepackage{tikz}
\usepackage{pgfplots}
\pgfplotsset{compat=1.18}
\usetikzlibrary{shapes.geometric, arrows, positioning}

% Setup (simplified without fancyhdr)
\pagestyle{plain}

% Custom commands (match energy-calculations.tex)
\newcommand{\code}[1]{\textcolor{blue}{\texttt{#1}}}
\newcommand{\highlight}[1]{\textcolor{blue}{\textbf{#1}}}

% Colors
\definecolor{primaryblue}{RGB}{0,102,204}
\definecolor{secondarygreen}{RGB}{0,153,76}
\definecolor{accentorange}{RGB}{255,102,0}

\begin{document}

% Start directly with content (no title page or TOC)
\section{Building Overview}

\subsection{Building Characteristics}
This analysis covers the conversion of {{PTAC_UNITS}} PTAC units to PTHP systems at {{ADDRESS}}, {{BOROUGH}}.

\textbf{Building Details:}
\begin{center}
\begin{tabular}{ll}
\hline
\textbf{Database Variable} & \textbf{Value} \\
\hline
\code{yearBuilt} & {{YEAR_BUILT}} \\
\code{stories} & {{STORIES}} \\
\code{buildingClass} & {{BUILDING_CLASS}} \\
\code{totalSquareFeet} & {{TOTAL_SQ_FT}} \\
\code{totalResidentialUnits} & {{TOTAL_UNITS}} \\
\code{ptacUnits} & {{PTAC_UNITS}} \\
\code{capRate} & {{CAP_RATE}}\% \\
\code{isRentStabilized} & {{IS_RENT_STABILIZED}} \\
\hline
\end{tabular}
\end{center}

\subsection{Unit Mix Breakdown}
\textbf{Source:} {{UNIT_BREAKDOWN_SOURCE}}

\begin{center}
\begin{tabular}{lc}
\hline
Unit Type & Count \\
\hline
Studio & {{STUDIO_UNITS}} \\
One Bedroom & {{ONE_BED_UNITS}} \\
Two Bedroom & {{TWO_BED_UNITS}} \\
Three+ Bedroom & {{THREE_PLUS_UNITS}} \\
\hline
\end{tabular}
\end{center}

% Section 2: Current PTAC Energy Analysis
\section{Current PTAC Energy Consumption}

\subsection{Heating Energy (Natural Gas)}
Annual building heating consumption:
\begin{align}
\text{Annual Therms} &= {{PTAC_UNITS}} \times 255 \text{ therms/unit} \\
&= {{ANNUAL_BUILDING_THERMS_PTAC}} \text{ therms} \quad \code{annualBuildingThermsHeatingPTAC}
\end{align}

\begin{align}
\text{Annual MMBtu} &= {{ANNUAL_BUILDING_THERMS_PTAC}} \times 0.1 \text{ MMBtu/therm} \\
&= {{ANNUAL_BUILDING_MMBTU_HEATING_PTAC}} \text{ MMBtu} \quad \code{annualBuildingMMBtuHeatingPTAC}
\end{align}

\subsection{Cooling Energy (Electricity)}
Annual building cooling consumption:
\begin{align}
\text{Annual kWh} &= {{PTAC_UNITS}} \times 16{,}000 \text{ kWh/unit} \\
&= {{ANNUAL_BUILDING_KWH_COOLING_PTAC}} \text{ kWh} \quad \code{annualBuildingkWhCoolingPTAC}
\end{align}

\begin{align}
\text{Annual MMBtu} &= {{ANNUAL_BUILDING_KWH_COOLING_PTAC}} \times 0.003412 \text{ MMBtu/kWh} \\
&= {{ANNUAL_BUILDING_MMBTU_COOLING_PTAC}} \text{ MMBtu} \quad \code{annualBuildingMMBtuCoolingPTAC}
\end{align}

\subsection{Total PTAC Energy}
\begin{align}
\text{Total MMBtu} &= {{ANNUAL_BUILDING_MMBTU_HEATING_PTAC}} + {{ANNUAL_BUILDING_MMBTU_COOLING_PTAC}} \\
&= {{ANNUAL_BUILDING_MMBTU_TOTAL_PTAC}} \text{ MMBtu}
\end{align}

\subsection{PTAC Energy Costs}
Annual operating costs:
\begin{align}
\text{Heating Cost} &= {{ANNUAL_BUILDING_THERMS_PTAC}} \times \$1.45/\text{therm} \\
&= \${{HEATING_COST_PTAC}} \quad \code{heatingCostPTAC}
\end{align}

\begin{align}
\text{Cooling Cost} &= {{ANNUAL_BUILDING_KWH_COOLING_PTAC}} \times \$0.24/\text{kWh} \\
&= \${{COOLING_COST_PTAC}} \quad \code{coolingCostPTAC}
\end{align}

\begin{align}
\text{Total PTAC Cost} &= \${{HEATING_COST_PTAC}} + \${{COOLING_COST_PTAC}} \\
&= \${{ANNUAL_BUILDING_COST_PTAC}} \quad \code{annualBuildingCostPTAC}
\end{align}

% Section 3: PTHP Energy Analysis
\section{PTHP System Energy Consumption}

\subsection{EFLH Calculation}
For this {{YEAR_BUILT}} building with {{STORIES}} stories, the Equivalent Full Load Hours (EFLH) is {{EFLH_HOURS}} hours \quad \code{eflhHours}.

\subsection{PTHP Heating Energy}
Using the EFLH methodology for accurate heating calculations:
\begin{align}
\text{Annual kWh Heating} &= \frac{8 \text{ KBtu}}{3.412 \text{ kW/KBtu}} \times \frac{1}{1.51 \text{ COP}} \times {{EFLH_HOURS}} \times {{PTAC_UNITS}} \\
&= {{ANNUAL_BUILDING_KWH_HEATING_PTHP}} \text{ kWh} \quad \code{annualBuildingkWhHeatingPTHP}
\end{align}

\begin{align}
\text{Annual MMBtu Heating} &= {{ANNUAL_BUILDING_KWH_HEATING_PTHP}} \times 0.003412 \\
&= {{ANNUAL_BUILDING_MMBTU_HEATING_PTHP}} \text{ MMBtu} \quad \code{annualBuildingMMBtuHeatingPTHP}
\end{align}

\subsection{PTHP Cooling Energy}
Cooling consumption remains the same as PTAC:
\begin{align}
\text{Annual kWh Cooling} &= {{ANNUAL_BUILDING_KWH_COOLING_PTHP}} \text{ kWh} \quad \code{annualBuildingkWhCoolingPTHP} \\
\text{Annual MMBtu Cooling} &= {{ANNUAL_BUILDING_MMBTU_COOLING_PTHP}} \text{ MMBtu} \quad \code{annualBuildingMMBtuCoolingPTHP}
\end{align}

\subsection{Total PTHP Energy}
\begin{align}
\text{Total MMBtu} &= {{ANNUAL_BUILDING_MMBTU_HEATING_PTHP}} + {{ANNUAL_BUILDING_MMBTU_COOLING_PTHP}} \\
&= {{ANNUAL_BUILDING_MMBTU_TOTAL_PTHP}} \text{ MMBtu} \quad \code{annualBuildingMMBtuTotalPTHP}
\end{align}

\subsection{PTHP Energy Costs}
All energy from electricity:
\begin{align}
\text{Total kWh} &= {{ANNUAL_BUILDING_KWH_HEATING_PTHP}} + {{ANNUAL_BUILDING_KWH_COOLING_PTHP}} \\
&= {{TOTAL_KWH_PTHP}} \text{ kWh} \quad \code{totalKWhPTHP}
\end{align}

\begin{align}
\text{Total PTHP Cost} &= {{TOTAL_KWH_PTHP}} \times \$0.24/\text{kWh} \\
&= \${{ANNUAL_BUILDING_COST_PTHP}} \quad \code{annualBuildingCostPTHP}
\end{align}

% Section 4: Energy Savings
\section{Energy Savings Analysis}

\subsection{Energy Reduction}
\begin{align}
\text{Energy Reduction} &= \frac{{{ANNUAL_BUILDING_MMBTU_TOTAL_PTAC}} - {{ANNUAL_BUILDING_MMBTU_TOTAL_PTHP}}}{{{ANNUAL_BUILDING_MMBTU_TOTAL_PTAC}}} \times 100\% \\
&= {{ENERGY_REDUCTION_PCT}}\% \quad \code{energyReductionPercentage}
\end{align}

\subsection{Annual Cost Savings}
\begin{align}
\text{Annual Savings} &= \${{ANNUAL_BUILDING_COST_PTAC}} - \${{ANNUAL_BUILDING_COST_PTHP}} \\
&= \${{ANNUAL_ENERGY_SAVINGS}} \quad \code{annualEnergySavings}
\end{align}

% Section 5: Retrofit Cost
\section{Retrofit Cost Analysis}

\begin{align}
\text{Unit Cost} &= (\$1{,}100 + \$450) \times 1.10 = \$1{,}705/\text{unit} \\
\text{Total Cost} &= {{PTAC_UNITS}} \times \$1{,}705 \\
&= \${{TOTAL_RETROFIT_COST}} \quad \code{totalRetrofitCost}
\end{align}

% Section 6: LL97 Compliance Analysis
\section{LL97 Compliance \& Carbon Reduction}

\subsection{Current Building Emissions}
Based on LL84 data, current emissions: {{TOTAL_BUILDING_EMISSIONS_LL84}} tCO₂e \quad \code{totalBuildingEmissionsLL84}

\subsection{Emissions Budgets by Period}
\begin{tabular}{lc}
\hline
Compliance Period & Budget (tCO₂e) \\
\hline
2024-2029 & {{EMISSIONS_BUDGET_2024_2029}} \\
2030-2034 & {{EMISSIONS_BUDGET_2030_2034}} \\
2035-2039 & {{EMISSIONS_BUDGET_2035_2039}} \\
2040-2049 & {{EMISSIONS_BUDGET_2040_2049}} \\
\hline
\end{tabular}

\subsection{BE Credit Analysis}
Beneficial Electrification credits from heating electrification:
\begin{align}
\text{BE Credit (2024-2027)} &= {{ANNUAL_BUILDING_KWH_HEATING_PTHP}} \times 0.0013 = {{BE_CREDIT_BEFORE_2027}} \text{ tCO₂e} \quad \code{beCreditBefore2027} \\
\text{BE Credit (2027-2029)} &= {{ANNUAL_BUILDING_KWH_HEATING_PTHP}} \times 0.00065 = {{BE_CREDIT_2027_2029}} \text{ tCO₂e} \quad \code{beCredit2027to2029}
\end{align}

\subsection{LL97 Fee Analysis}
\textbf{Without Upgrade (Annual Fees):}
\begin{itemize}
\item 2024-2029: \${{ANNUAL_FEE_NO_UPGRADE_2024_2029}}
\item 2030-2034: \${{ANNUAL_FEE_NO_UPGRADE_2030_2034}}
\item 2035-2039: \${{ANNUAL_FEE_NO_UPGRADE_2035_2039}}
\item 2040-2049: \${{ANNUAL_FEE_NO_UPGRADE_2040_2049}}
\end{itemize}

\textbf{With Upgrade \& BE Credits (Annual Fees):}
\begin{itemize}
\item Before 2027: \${{ADJUSTED_ANNUAL_FEE_BEFORE_2027}}
\item 2027-2029: \${{ADJUSTED_ANNUAL_FEE_2027_2029}}
\item 2030-2034: \${{ADJUSTED_ANNUAL_FEE_2030_2034}}
\item 2035-2039: \${{ADJUSTED_ANNUAL_FEE_2035_2039}}
\item 2040-2049: \${{ADJUSTED_ANNUAL_FEE_2040_2049}}
\end{itemize}

% Section 7: Financial Analysis
\section{Financial Analysis}

\subsection{Payback Analysis}
Simple payback period: {{PAYBACK_PERIOD}} \quad \code{simplePaybackPeriod}

\subsection{Loan Analysis}
\textbf{Loan Terms:}
\begin{itemize}
\item Principal: \${{TOTAL_RETROFIT_COST}}
\item Term: 15 years
\item Interest Rate: 6.0\%
\item Monthly Payment: \${{MONTHLY_PAYMENT}}
\item Total Interest: \${{TOTAL_INTEREST_PAID}}
\end{itemize}

\subsection{LL97 Fee Avoidance}
\textbf{Annual Fee Avoidance:}
\begin{itemize}
\item 2024-2027: \${{ANNUAL_LL97_FEE_AVOIDANCE_2024_2027}}
\item 2027-2029: \${{ANNUAL_LL97_FEE_AVOIDANCE_2027_2029}}
\item 2030-2034: \${{ANNUAL_LL97_FEE_AVOIDANCE_2030_2034}}
\item 2035-2039: \${{ANNUAL_LL97_FEE_AVOIDANCE_2035_2039}}
\item 2040-2049: \${{ANNUAL_LL97_FEE_AVOIDANCE_2040_2049}}
\end{itemize}

% Section 8: NOI Impact
\section{Net Operating Income Impact}

\subsection{Current Building NOI}
Estimated current NOI: \${{ANNUAL_BUILDING_NOI}} \quad \code{annualBuildingNOI}

\textbf{Rent Stabilized Status:} {{IS_RENT_STABILIZED_DISPLAY}} \quad \code{isRentStabilized}

\subsection{NOI Enhancement}
Operating expense savings from reduced energy costs contribute to NOI improvement.

% Section 9: Financial Projections
\section{Financial Projections}

\subsection{Loan Balance vs Savings}
\begin{center}
\begin{tikzpicture}
\begin{axis}[
    title={\textbf{Loan Balance vs. Cumulative Savings}},
    xlabel={Year},
    ylabel={Amount},
    xmin=2025, xmax=2042,
    ymin={{LOAN_CHART_Y_MIN}}, ymax={{LOAN_CHART_Y_MAX}},
    legend pos=north east,
    grid=major,
    grid style={line width=0.1pt, draw=gray!30},
    major grid style={line width=0.2pt, draw=gray!50},
    width=14cm,
    height=10cm,
    axis background/.style={fill=gray!5},
    scaled y ticks=false,
    y tick label style={/pgf/number format/fixed,/pgf/number format/1000 sep=\,},
    yticklabel={\$\pgfmathprintnumber{\tick}},
    xtick={2025,2026,2027,2028,2029,2030,2031,2032,2033,2034,2035,2036,2037,2038,2039,2040,2041,2042},
    x tick label style={rotate=90, anchor=east, /pgf/number format/fixed,/pgf/number format/1000 sep=},
    xticklabels={\textbf{2025},2026,2027,2028,2029,2030,2031,2032,2033,2034,2035,2036,2037,2038,2039,2040,2041,2042}
]

% Loan balance with proper amortization curve
\addplot[color=blue!70!black, line width=2pt, smooth] coordinates {
    {{LOAN_BALANCE_COORDINATES}}
};

% Cumulative savings - no savings in 2025 (upgrade year), savings begin in 2026
\addplot[color=green!60!black, line width=2pt, smooth] coordinates {
    {{CUMULATIVE_SAVINGS_COORDINATES}}
};

% Add marker at payback point  
\addplot[only marks, mark=*, mark size=4pt, color=red] coordinates {({{PAYBACK_PERIOD}}, {{TOTAL_RETROFIT_COST}})};

% Add professional annotation with arrow - use template variable
\node[anchor=south east, align=center, font=\small\sffamily] at (axis cs:2036,{{LOAN_CHART_ANNOTATION_Y}}) {
    \textcolor{black}{\textsc{Simple Payback}}\\
    \textcolor{black}{\textsc{Period Reached}}
};

% Upgrade completion vertical line and label
\draw[color=blue, line width=2pt, dashed] (axis cs:2025,0) -- (axis cs:2025,{{LOAN_CHART_Y_MAX}});
\node[anchor=south, font=\scriptsize\sffamily, rotate=90, color=blue] at (axis cs:2025,{{LOAN_CHART_LABEL_Y}}) {Upgrade Completed};

\legend{Remaining Loan Balance, Cumulative Savings}

\end{axis}
\end{tikzpicture}
\end{center}

\textbf{Savings and Loan Balance Data by Year:}
\begin{center}
\begin{tabular}{lccc}
\hline
Year & Annual Savings & Cumulative Savings & Loan Balance \\
\hline
{{CUMULATIVE_SAVINGS_DATA_TABLE}}
\hline
\end{tabular}
\end{center}

\subsection{NOI Impact by Scenario}
\begin{center}
\begin{tikzpicture}
\begin{axis}[
    title={\textbf{Net Operating Income by Scenario}},
    xlabel={Year},
    ylabel={NOI (\$Millions)},
    xmin=2025, xmax=2048,
    ymin={{NOI_Y_MIN}}, ymax={{NOI_Y_MAX}},
    legend pos=south east,
    legend style={draw=black, fill=white, fill opacity=0.8, text opacity=1},
    grid=major,
    grid style={line width=0.1pt, draw=gray!30},
    major grid style={line width=0.2pt, draw=gray!50},
    width=14cm,
    height=10cm,
    axis background/.style={fill=gray!5},
    scaled y ticks=false,
    y tick label style={/pgf/number format/fixed,/pgf/number format/1000 sep=\,},
    yticklabel={\$\pgfmathprintnumber{\tick}M},
    xtick={2025,2026,2028,2030,2032,2034,2036,2038,2040,2042,2044,2046,2048},
    x tick label style={rotate=90, anchor=east, /pgf/number format/fixed,/pgf/number format/1000 sep=},
    xticklabels={2025,2026,2028,2030,2032,2034,2036,2038,2040,2042,2044,2046,2048}
]

% NOI with PTHP upgrade - enhanced NOI from energy savings
\addplot[color=green!60!black, line width=3pt, smooth] coordinates {
    {{NOI_WITH_UPGRADE_COORDINATES}}
};

% NOI without upgrade - reduced NOI due to LL97 fees
\addplot[color=red!70!black, line width=3pt, smooth] coordinates {
    {{NOI_NO_UPGRADE_COORDINATES}}
};

% Period markers
\draw[color=blue, line width=2pt, dashed] (axis cs:2025,{{NOI_Y_MIN}}) -- (axis cs:2025,{{NOI_Y_MAX}});
\draw[color=orange, line width=1.5pt, dashed] (axis cs:2030,{{NOI_Y_MIN}}) -- (axis cs:2030,{{NOI_Y_MAX}});
\draw[color=orange, line width=1.5pt, dashed] (axis cs:2035,{{NOI_Y_MIN}}) -- (axis cs:2035,{{NOI_Y_MAX}});

% Upgrade completion label
\node[anchor=south, font=\scriptsize\sffamily, rotate=90, color=blue] at (axis cs:2025,{{NOI_Y_MAX}}) {Upgrade Completed};

% Period labels  
\node[anchor=south, font=\scriptsize\sffamily, rotate=90] at (axis cs:2030,{{NOI_Y_MAX}}) {Higher Penalties};
\node[anchor=south, font=\scriptsize\sffamily, rotate=90] at (axis cs:2035,{{NOI_Y_MAX}}) {Maximum Penalties};

\legend{NOI with Upgrade, NOI without Upgrade}

\end{axis}
\end{tikzpicture}
\end{center}

\textbf{NOI Data by Year (\$Millions):}
\begin{center}
\begin{tabular}{lcc}
\hline
Year & NOI with Upgrade & NOI without Upgrade \\
\hline
{{NOI_DATA_TABLE}}
\hline
\end{tabular}
\end{center}

\subsection{Property Value Enhancement}
\begin{center}
\begin{tikzpicture}
\begin{axis}[
    title={\textbf{Property Value by Scenario}},
    xlabel={Year},
    ylabel={Property Value (\$Millions)},
    xmin=2025, xmax=2040,
    ymin={{PROPERTY_VALUE_Y_MIN}}, ymax={{PROPERTY_VALUE_Y_MAX}},
    legend pos=south east,
    legend style={draw=black, fill=white, fill opacity=0.8, text opacity=1},
    grid=major,
    grid style={line width=0.1pt, draw=gray!30},
    major grid style={line width=0.2pt, draw=gray!50},
    width=14cm,
    height=10cm,
    axis background/.style={fill=gray!5},
    scaled y ticks=false,
    y tick label style={/pgf/number format/fixed,/pgf/number format/1000 sep=\,},
    yticklabel={\$\pgfmathprintnumber{\tick}M},
    xtick={2025,2026,2027,2028,2029,2030,2031,2032,2033,2034,2035,2036,2037,2038,2039,2040},
    x tick label style={rotate=90, anchor=east, /pgf/number format/fixed,/pgf/number format/1000 sep=},
    xticklabels={\textbf{2025},2026,2027,2028,2029,2030,2031,2032,2033,2034,2035,2036,2037,2038,2039,2040}
]

% Property value with PTHP upgrade - upgrade completes in 2025, value increases in 2026
\addplot[color=green!60!black, line width=3pt, smooth] coordinates {
    {{PROPERTY_VALUE_WITH_UPGRADE_COORDINATES}}
};

% Property value without upgrade - dramatic drop in 2026 when LL97 fees kick in
\addplot[color=red!70!black, line width=3pt, smooth] coordinates {
    {{PROPERTY_VALUE_NO_UPGRADE_COORDINATES}}
};

% Period markers
\draw[color=blue, line width=2pt, dashed] (axis cs:2025,{{PROPERTY_VALUE_Y_MIN}}) -- (axis cs:2025,{{PROPERTY_VALUE_Y_MAX}});
\draw[color=orange, line width=1.5pt, dashed] (axis cs:2030,{{PROPERTY_VALUE_Y_MIN}}) -- (axis cs:2030,{{PROPERTY_VALUE_Y_MAX}});
\draw[color=orange, line width=1.5pt, dashed] (axis cs:2035,{{PROPERTY_VALUE_Y_MIN}}) -- (axis cs:2035,{{PROPERTY_VALUE_Y_MAX}});

% Upgrade completion label
\node[anchor=south, font=\scriptsize\sffamily, rotate=90, color=blue] at (axis cs:2025,{{PROPERTY_VALUE_Y_MAX}}) {Upgrade Completed};

% Period labels
\node[anchor=south, font=\scriptsize\sffamily, rotate=90] at (axis cs:2032,{{PROPERTY_VALUE_Y_MAX}}) {Higher LL97 Penalties};
\node[anchor=south, font=\scriptsize\sffamily, rotate=90] at (axis cs:2037.5,{{PROPERTY_VALUE_Y_MAX}}) {Continued LL97};

% Highlight the value gap - use template variables for positioning
\fill[yellow!30, opacity=0.5] (axis cs:2030,{{PROPERTY_VALUE_GAP_Y_MIN}}) rectangle (axis cs:2040,{{PROPERTY_VALUE_GAP_Y_MAX}});
\node[anchor=center, font=\small\sffamily, color=black] at (axis cs:2035,{{PROPERTY_VALUE_GAP_CENTER}}) {\textbf{Property Value Gap}};
\node[anchor=center, font=\small\sffamily, color=black] at (axis cs:2035,{{PROPERTY_VALUE_GAP_LABEL}}) {\textbf{\${{NET_PROPERTY_VALUE_GAIN}}}};

\legend{Property Value with Upgrade, Property Value without Upgrade}

\end{axis}
\end{tikzpicture}
\end{center}

\textbf{Property Value Data by Year (\$Millions):}
\begin{center}
\begin{tabular}{lcc}
\hline
Year & Value with Upgrade & Value without Upgrade \\
\hline
{{PROPERTY_VALUE_DATA_TABLE}}
\hline
\end{tabular}
\end{center}

% Summary
\section{Executive Summary}

This analysis demonstrates the financial and environmental benefits of converting {{PTAC_UNITS}} PTAC units to PTHP systems at {{ADDRESS}}.

\textbf{Key Results:}
\begin{itemize}
\item \textbf{Energy Reduction:} {{ENERGY_REDUCTION_PCT}}\% annual energy use reduction
\item \textbf{Annual Savings:} \${{ANNUAL_ENERGY_SAVINGS}} in energy costs
\item \textbf{Payback Period:} {{PAYBACK_PERIOD}} years
\item \textbf{LL97 Compliance:} Significant fee avoidance across all periods
\item \textbf{Property Value:} \${{NET_PROPERTY_VALUE_GAIN}} value enhancement
\end{itemize}

The retrofit investment of \${{TOTAL_RETROFIT_COST}} provides strong returns through energy savings, LL97 fee avoidance, and property value enhancement.

\end{document}